\documentclass[a4paper,portrait,12pt]{article}
\usepackage[top=0.5in]{geometry}
\usepackage{hyperref}
\usepackage{amssymb}
\usepackage{fullpage}
\usepackage{epstopdf}
\usepackage{float}
\usepackage{fancybox}
\usepackage{tikz}
\usepackage{subfloat}
\usepackage{subcaption}
\usepackage{color}
\usepackage[utf8]{inputenc}
\usepackage{graphicx}
\usepackage{amsmath}
\usepackage{latexsym}
\usepackage{amsthm}
\usepackage{eucal}
\usepackage{eufrak}
\usepackage{subfiles}
\usepackage{listings}
\usepackage{verbatim}
\usepackage{csquotes}
\usepackage{program}
\usepackage{mathtools}

\theoremstyle{definition}
\newtheorem{definition}{Definizione}[section]

\newtheorem{proposition}{Proposizione}
\newtheorem{corollary}{Corollario}

\providecommand{\abs}[1]{\lvert#1\rvert}

\DeclarePairedDelimiter{\ceil}{\lceil}{\rceil}
\DeclarePairedDelimiter{\floor}{\lfloor}{\rfloor}
 
\begin{document}

\title{Appunti Dati e Algoritmi 2}

\maketitle
\date
\newpage

\tableofcontents
\newpage


\section{Teoria dell'NP-Completezza}


\subsection{Introduzione} 

Le \textit{classi di complessità} sono insiemi di problemi. Le principali classi di complessità 
note sono:
\begin{enumerate}
\item problemi che ammettono algoritmi \textit{efficienti} (ovvero polinomiali, di qualunque grado);
\item problemi che ammettono limite di esecuzione inferiore del tipo $f(n) = c^n$ con $c > 1$, sono un numero ristretto
	di problemi, perlopiù costruiti appositiamente per avere tale limite;
\item problemi che non ammettono algoritmo di risoluzione, ovvero problemi \textit{indecidibili} come l'\textit{Halting 
	Problem} di Turing \footnote{Il problema originale è in realtà il famoso \textit{Entscheidungsproblem}, ovvero 
	\textit{il problema della decisione}, posto da Hilbert nel 1928 e risolto contemporaneamente nel 1936 da A. Turing
 	(con le macchine omonime) e da A. Church (per mezzo del \textit{lambda calcolo})};
\item problemi per i quali non è stato determinato un algorimo efficiente ma di cui non si ha nemmeno un \textit{lower 
	bound}, chiamati per questo \textit{problemi intrattabili}; si tratta di problemi reali, riscontrabili in 
	moltissimi ambiti scientifici e ingegneristici; possiedono inoltre un'interessante \textit{proprietà di 
	chiusura}: risolto uno di questi problemi in tempo polinomiale è possibile risolvere in modo simile anche tutti
	gli altri problemi di questa classe.
\end{enumerate} 

Per semplificare la teoria ci si occupa soltanto di \textit{problemi decisionali}, quindi problemi nella forma
\begin{equation}
\Pi : I \mapsto \{yes,no\}
\end{equation}
che differiscono da quelli precedentemente incontrati nella forma $\Pi \subseteq I \times S$.\\ 
Sia $\Pi_C (x)$ il problema concreto tale che
\begin{equation}
\forall x \in \{0,1\}^*,\quad \Pi_C (x) = 
	\left\{
		\begin{array}{ll}
			1 \quad se\quad (\exists i \in I\quad |\quad e(i) = x)\quad\wedge\quad(\Pi_A (i) = x)\\
			0 \quad altrimenti\\
		\end{array}
	\right.
\end{equation}
si definisce la \textit{complessità} (o il tempo) associato ad un algoritmo che risolva $\Pi_C$ la funzione
\begin{equation}
T_{A_{\Pi_C}} (n) = max\{\# \quad passi\quad eseguiti\quad da\quad A_{\Pi_C} (x) \quad\forall x \quad | \quad 
\left|x\right| = n \}
\end{equation}
dove $\left|x\right| = \left|e(i)\right|$ è la taglia di $x$ data dalla funzione di encoding $e$.\\
D'ora in poi si utilizzeranno encoding \textit{concisi}, più vicini possibile all'encoding di taglia minima così da
poter avere problemi ugulmente definiti come $\Pi_C : \{0,1\}^* \mapsto \{0,1\}$.\\


\subsection{Linguaggi Formali} 

Sia $\Sigma = {0,1}$ un alfabeto, si definisce \textit{linguaggio formale} $L$ su $\Sigma$
un insieme di stringhe $L \subseteq \Sigma^*$, $L \subseteq {0,1}^*$. Esempi:
\begin{itemize}
	\item $\epsilon$, stringa vuota
	\item $L = \{\emptyset\}$, linguaggio vuoto
	\item $L^0 = L = \{\epsilon\}$, linguaggio contenente soltanto la stringa vuota
	\item $L_1 \cup L_2$, $L_1 \cap L_2$, unione e intersezione di linguaggi
	\item $L^C = \{0,1\}^* - L$, linguaggio complementare
	\item $L_1\cdot L_2 = \{z \in {0,1}^* \quad | \quad \exists x \in L_1,\quad \exists y \in L_2 \quad | 
		\quad z = \langle x,y \rangle\}$, concatenazione di linguaggi
	\item $L^* = \bigcup_{i = 0}^{\infty} L^i$, tutte le concatenazioni finite di tutti gli elementi di $L$.
\end{itemize}
Dato un problema concreto $\Pi_C : \{0,1\}^* \mapsto \{0,1\}$, si definisce
\begin{equation}
L_{\Pi_C} = \{x \in \{0,1\}^* \quad | \quad \Pi_C (x) = 1\}
\end{equation}
è il linguaggio accettato dal problema concreto $\Pi_C$.\\
Fino a qui si è sviluppata una teoria riguardante soltamente i problemi decisionali, mentre la maggior parte dei problemi
sono problemi di ottimizzazione, ovvero nella forma
\begin{equation}
\Pi_{OPT} \subseteq I \times S, \quad c : S \mapsto \mathbb{R}^+ \cup \{0\}\\
\forall i \in I \quad determina \quad s^*\quad |\quad (i\Pi s^*) \quad\wedge\quad c(s^*) = inf/sup\{c(s),i\Pi s\}
\end{equation}
Tuttavia gli algoritmi che risolvono problemi decisionali possono essere introdotti in procedure che risolvono problemi 
di ottimizzazione.
Sia $A$ un algoritmo con input in $\{0,1\}^*$ e output in $\{0,1\}$, $A$ \textit{accetta} $x \in \{0,1\}^*$ 
se $A(x) = 1$, $A$ \textit{rigetta} $x$ se $A(x) = 0$ ($A$ termina in un numero finito di passi).
Si dice \textit{linguaggio accettatto} da $A$ il linguaggio definito come
\begin{equation}
L_A = \{x \in \{0,1\}^* \quad | \quad A(x) = 1\}
\end{equation}
Un linguaggio $L$ è deciso in tempo polinomiale da un algoritmo $A$ se
\begin{enumerate}
	\item $L = L_A$
	\item $\exists k_1 \in \mathbb{N} \quad | \quad T_A(\left|x\right|) = O(\left|x\right|^{k_1})$
\end{enumerate}


\subsection{Classi di complessità} 

Si definisce la classe $\mathcal{P}$ la classe
\begin{equation}
\mathcal{P} = \{L \subseteq \{0,1\} \quad | \quad L \quad deciso \quad in \quad tempo \quad polinomiale\} 
\end{equation}
\begin{equation}
= \{\Pi_C : \{0,1\}^* \mapsto \{0,1\} \quad | \quad L \quad deciso \quad in \quad tempo \quad polinomiale\}
\end{equation}
poiché esiste un isomorfismo tra i problemi concreti $\Pi_C$ e i linguaggi $L$.\\

Per introdurre la classe $\mathcal{NP}$ è necessario comprendere la differenza tra i concetti di
\begin{itemize}
	\item \textit{risoluzione}, dato un problema $\Pi \subseteq I\times S$, l'algoritmo che lo risolve riceve 
	un'istanza $i \in I$ e restituisce una soluzione $s \in S$;
	\item \textit{verificabilità}, dato un problema $\Pi \subseteq I\times S$, l'algoritmo riceve 
	un'istanza de problema $i \in I$ e una sua presunta soluzione $q$ (chiamata \textit{certificato}) e verifica 
	che $q \in S$ o meno. 
\end{itemize}
Sia $V(x,y)$ un algoritmo verificatore. Si dice che $x$ è verificato da $V$ se $\exists y \quad | \quad V(x,y) = 1$. Si
dice che $L$ è verificato in tempo polinomiale se $\exists V(x,y)$ tale che:
\begin{enumerate}
	\item $L = L_V$;
	\item $\exists c_1,k_1 \in \mathbb{R}^+\cup \{0\} \quad | \quad \forall x \in L \quad \exists y \quad 
	V(x,y) = 1$ e $\left|y\right| \le c_1 \left|x\right|^{k_1}$
	\item $\exists c_2,k_2 \in \mathbb{R}^+\cup \{0\} \quad | T(\left|x\right|,\left|y\right|) \le 
	c_2 (\left|x\right| + \left|y\right|)^{k_2}$
\end{enumerate}
Come si può vedere, al punto 2 viene posta una condizione sulla dimensione del certificato. Se questa non fosse presente
si potrebbe pensare di fornire tutte le possibili soluzioni come certificato. Ma questo equvarrebbe a risolvere il
problema con la forza bruta.\\

La classe $\mathcal{NP}$ è quindi definita come
\begin{equation}
\mathcal{NP} = \{L \subseteq \{0,1\}^* \quad |\quad L \quad verificabile \quad in \quad tempo \quad polinomiale\}
\end{equation}
\textbf{Nota bene.} $\mathcal{NP}$ non significa \textit{Non Polynomial time}, ma bensì \textit{Nondeterministic 
Polynomial time}; questo perché la classe $\mathcal{NP}$ può essere alternativamente definita per mezzo delle
macchine di Turing non deterministiche.\\

Con questo teorema si rende esplicito il rapporto tra le classi $\mathcal{P}$ e $\mathcal{NP}$.\\
\textbf{Teorema.} $\mathcal{P} \subseteq \mathcal{NP}$, ovvero $L \subseteq \{0,1\}^* \in \mathcal{P} \Rightarrow 
L \in \mathcal{NP}$.\\

\textbf{Dimostrazione.} Intuitivamente, essendo $\mathcal{P} = {L \quad | \quad risolubile \quad itp}$ e 
$\mathcal{NP} = {L \quad | \quad verificabile \quad itp}$, se $L \in \mathcal{P} \Rightarrow \exists A_L$ che
decide $L$ in tempo polinomiale. Quindi, se $L \in \mathcal{NP}$ deve esistere $V_L$ algoritmo verificatore che
verifica $L$ in tempo polinomiale (\textit{aggiungere dimostrazione formale}).\\

\textbf{Riducibilità.} Sia $\Pi_1 : \{0,1\}^* \mapsto \{0,1\}$ e $\Pi_2 : \{0,1\}^* \mapsto \{0,1\}$ e sia $f$ la 
funzione $f : \Pi_1 \mapsto \Pi_2$, $f : i_1 \mapsto i_2 = f(i_1)$. In questo modo abbiamo che $\Pi_1 (i_1) = 
\Pi_2 (f(i_1)) = \Pi_2 (i_2)$.\\
Questa funzione è calcolabile in tempo polinomiale se $\exists A_f : \{0,1\}^* \mapsto {0,1} \quad | \quad 
A_f(x) = f(x)$ e $\exists k > 0 \quad | \quad T_{A_f} (\left|x\right|) = O(\left|x\right|^k)$.

\end{document}
